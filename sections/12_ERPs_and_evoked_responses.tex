% TODO: insert images!
\subsection{Preprocessing}
Whenever data are recorded during an experiment, independently on the aquisition
system, it is necessary to perform some preprocessing tasks. This is done in order to
remove the residual noise before performing the actual analysis on the data.\\
The steps commonly carried out during preprocessing are (not necessarily in
this order):
\begin{itemize}
    \item Filtering
    \item Down-sampling
    \item Epoching
    \item Cleaning
    \item Artifacts rejection
    \item Re-referencing (mainly related to EEG recordings) 
\end{itemize}
\subsubsection{Filtering}
Filters are employed to remove unwanted frequencies, such as the \(50\,Hz\) due to
the electrical power supply. Notice that filtering brings some issues as well, as it
affects the overall data.\\
Several types of filters do exist:
\begin{itemize}
    \item \textbf{High-pass:} remove the DC offset and slow trends in the data.
    \item \textbf{Low-pass:} remove high-frequency noise, performing an action similar to
    smoothing.
    \item \textbf{Notch (a.k.a. band-stop):} remove the power line noise and its harmonics,
    it is a sort of extremely narrow band-pass filter.
    \item \textbf{Band-pass:} it allows only a range of frequencies and removes the others.
    It is often obtained as a combination of high-pass and low-pass filters.
\end{itemize}
Some of the most important remarks about filters are reported below:
\begin{itemize}
    \item Generally, notch filters have very low ranges of allowed frequencies and
    this might create artifacts and affect the signal in a negative way. Notice
    that by increasing the range of a band-stop filter, the power of the filter itself
    is reduced, leading to a worse removal of noise.
    \item IIR (infinite-impulse response) filters and FIR (finite-impulse response)
    filters are both valuable choices; however, FIR filters are commonly preferred
    whenever the phase distortion is an issue, as they do not introduce it, wheres
    IIR filters might. Therefore, FIR filers ensures a better phase-time relationship.
    \item In case of band-pass filters, it is critical to properly select the
    bandwidth of interest: in general broadband filters are preferred for neural
    signal in a high-frequency range, especially \(>30\,Hz\), as the activity
    is non-purely oscillatory, such as the multi-unit activity of neurons. When
    interested in low-frequency oscillations - i.e. alpha band or theta band - it is
    more effective to apply a narrowband filter.
    \item Additionally, Morlet filtering is to be preferred when analysing the phase
    of multiple signals is being investigated, as it employes narrowband filters, it
    allows a better reconstruction of the oscillation phase.
\end{itemize}
\subsubsection{Down-sampling}
After filtering, it might be useful to down-sample the signal in order to reduce
its size, decreasing the computational load during the analysis. A critical issue to
take into account when down-sampling the signal is the Nyquist limit, such that
aliasing is avoided.
\subsubsection{Epoching}
In many types of analysis it is common to divide the signal in equally long
non-overlapping chunks, which are denominated epochs. If these chunks are taken
by a common type of event, then they are denominated trials. Generally:
\begin{itemize}
    \item \textbf{Epochs:} partitions of the signal not related to a particular event.
    \item \textbf{Trials:} partitions of a signal referred to a specific event.
\end{itemize}
In both cases, the signal is divided into subsequent time windows with a fixed
length.\\
A noticeable issue related to epoching is that by taking a particular portion of
the signal corresponds to multiplying the overall signal for a rectangular window.
This introduces edges at the extremities of the time window, which possibly results
in artifacts when filtering the epoch. Such an issue is particularly concerning when
the selected time window is short, due to the fact that the number of edges
(and consequently of artifacts) increases dramatically.\\
A possible solution might consist in performing filtering before epoching, however
in some types of analysis this might not be pratical nor feasible, therefore two
distinct strategies exist to deal with the filtering of epochs:
\begin{itemize}
    \item Epochs might be created wider than the signal of interest, therefore
    artifacts might be generated at the extremities of the epoched signal, but the
    centered portion of the signal, the one of interest, would not be affected.
    \item Alternatively, the epoched signal might be relicated symmetrically both
    before and after itself (reflected signal). Then, the analysis might be performed
    without generating artifacts on the initial epoched signal. Finally, the reflected
    portions of the initial signal can be once again discarded.
\end{itemize}
\subsubsection{Cleaning}
EEG and in general LFP recording techniques are much more prone to noise than
invasive technologies. In particular, other physiological activites might concur
in the noise, as they involve electrical activity as well:
\begin{itemize}
    \item \textbf{Blinking:} opening and closing the eyes generates clearly visible dipoles
    in the signal. This issue is especially observable in the channels referred to
    electrodes placed in the frontal part of the head, close to the eyes.
    \item \textbf{Saccadic eye movements:} rectangular-shaped artifacts are often found as a
    result of large saccades. Small saccades are generally not visible in EEG
    recordings. These artifacts are pretty easy to be identifies, due to their
    peculiar spatial distribution and duration.
    \item \textbf{EMG:} muscular activity is driven by electrical signals and they are often
    recorded by the EEG, especially when the muscles involved are the facial ones.
    An issue related to the EMG signals is that their frequency content is very similar
    to the one of the neural system, therefore they cannot be removed by a
    frequency-selective filtering.
    \item \textbf{EKG:} the electrical activity of the beating heart is larger in magnitude
    if compared to neural signals. These artifacts are characterized by a
    triangular shape and cannot be easily identified. Notice that also the flow of
    blood in the brain arteries is a source of noise.
    \item \textbf{Skin potential:} the impedance of the skin cannot be controlled.
    \item \textbf{Blocking:} the signal amplifier saturates, therefore it is not possible
    to properly record the signal, which instead becomes flat.
\end{itemize}
\subsubsection{Artifacts Rejection}
% TODO: write
\subsubsection{Re-referencing}
% TODO: write

\subsection{Event-Related Potentials}
% TODO: write