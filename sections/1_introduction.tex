The application of a stimulus to the nervous system leads to changes 
determined by two main properties:
\begin{itemize}
    \item \textbf{Excitability:} the capability of a nerve cell to react 
to an incoming impulse.
    \item \textbf{Plasticity:} certain permanent functional 
transformations arise in systems of neurons as a result of appropriate 
stimuli or their combination (it describes the possibility of the brain of 
maintaining a specific response over long time scales).
\end{itemize}
The communication between neurons occurs through action potentials, as 
shown in the following picture.
\begin{figure}[H]
    \includegraphics[scale=0.3]{1_1}
    \centering
\end{figure}
Measuring, analyzing, and processing brain signals is done for several 
reasons, but the three main ones are:
\begin{itemize}
    \item \textbf{Learn} more about the brain functioning.
    \item \textbf{Mimic} the brain functions in an artificial way to 
create new types of technologies - i.e. neuromorphic engineering.
    \item \textbf{Employ} brain signals for control and communication, as 
in the case of prosthetics.
\end{itemize}
This is actually what neuroengineering is about. It can be described as a 
tale of two loops: from one point of view, we have the development of new 
technologies from the knowledge of the brain, while from the other one, 
new technologies can be useful to treat the brain or the nervous system.\\

How do we interact with the brain?
Reading the neural code can be done through different levels of 
interaction of interfaces with the brain, and, depending on the level, 
different signals can be obtained. 
\begin{itemize}
    \item EEG level
    \item Epidural level
    \item Intracortical level
    \item Depth level
\end{itemize}
The neural code can either be read or written by exploiting several 
distinct techniques exhibiting different levels of:
\begin{align*}
    \begin{matrix}
        \textbf{Invasiveness}       &  & \textbf{Risk}                &  &
        \textbf{Spatial resolution} &  & \textbf{Temporal resolution}
    \end{matrix}
\end{align*}
The technology nowadays allows fair degrees of spatial and temporal 
resolution, with
the main reading techniques being fMRI, PET, optical imaging, EEG, ECoG, 
MEA, single neuron, and others. Notice that also the number of reading 
sites is constantly increasing, due to the
exploitation of multiple electrode devices.
\begin{figure}[H]
    \includegraphics[scale=0.3]{1_2}
    \centering
\end{figure}
The multi-scale brain model presented above aims at showing the different 
levels at which the brain activity can be investigated, in terms of 
spatial, temporal, and topological scales.
Let's stress once more that the brain can be studied at different level of 
complexity, which are somehow related to the risk due to the techniques 
involved to do so.
\begin{figure}[H]
    \includegraphics[scale=0.3]{1_3}
    \centering
\end{figure}
Notice that in general the functional unit of the brain is not a single 
neuron: it has soon been realized that the neurons, for the complexity of 
the complete structure, couldn't be studied only in isolation, but should 
also be analyzed in the different cell assemblies that they form.\\
There are many ways to study these structures, but in principle we can 
rely on two approaches: the \textit{in vitro} and \textit{in vivo} ones. 
Generally, this second family is more complex to employ, as it requires 
not to damage the brain of
the experiment subject. 

\subsection{In Vitro Systems}
One of the principal technologies to study \textit{in vitro} networks is 
\textbf{MEAs (Micro Electrode Arrays)}, which are able to detect the 
activity of the network at several recording sites and for extended time 
periods.\\
They are composed by a complex integrated microelectrode system (with 
typically 60 sites), that leads to a multi-site long-term stimulation and 
recording.
However, an issue related to MEAs is the difficulty often encountered in 
analysing the recorded data
and interpreting the results in a meaningful way.\\
We can use this device to record multiple types of signals, but the ones 
in which we are interested are mostly two:
\begin{itemize}
    \item \textbf{Local field potentials}, that can be recorded by 
recording the signal far from the neural network. They consist in the low 
frequency components (200/300 Hz) accounting for several neurons as a sort 
of superposition of their effects.
    \item \textbf{Multi-unit activity}, which can be extracted by 
measuring a signal in the vicinity of a neuron.  In general, a signal made 
of spikes is obtained by recording neural signals at high frequencies (300 
Hz-3 kHz), indicating the firing of a neuron. It is mostly related to the 
activity of one neuron but can also be influenced by the other ones 
nearby.
\end{itemize}

\textbf{Networks of neurons} represent an intermediate level of 
organization of the nervous system. 
In vitro neural networks are quite easy to obtain, and they are important, 
since they constitute a valuable experimental model for studying the 
collective electrophysiological and functional properties of the brain. In 
fact, the fundamental unit of the nervous system is not the single neuron 
but the \textbf{cell assembly}.\\

Neurons dissociated from the original tissue can be cultured in vitro from 
days up to months. During the in-vitro development, synapses start to be 
established and neurons begin to communicate among them, building a 2-D 
network. Networks freely develop, without barriers, and the basic 
electrophysiological, biochemical and pharmacological properties are 
maintained similar to those of in vivo neurons.\\
These cultures are mostly obtained from animals, even if new technologies 
are developing.\\
The processed that is followed is:
\begin{itemize}
    \item Pre-culture preparations:
    
    \begin{itemize}
        \item Sterilize the culturing substrate (e.g. set of 
microelectrodes arrays)
        \item Pre-treat the insulation layer (silicon nitride) with 
adhesion factors (Poli-L/D-Lysine and Laminin) in order to improve the 
coupling between neurons and electrodes.
    \end{itemize}
    
    \item Dissection:
    
    \begin{itemize}
        \item Pregnant rats at the gestation day 18-19 are usually used. 
        \item The pups are removed and their cortex isolated.
    \end{itemize}
    
    \item Dissociation:
    
    \begin{itemize}
        \item Add trypsin.
        \item Mechanical dissociation by using a Pasteur pipette and cell 
count.
        \item Suspend the cells in neurobasal medium and place them on the 
pre-treated substrate (final concentration about 50.000 cells/device).
    \end{itemize}
    
    \item Maintenance of the culture:
    
    \begin{itemize}
        \item Place the substrates in a humified incubator containing 0.5 
CO2 at 37°C
        \item Replace part of the old media with fresh media (neurobasal 
medium) each week
    \end{itemize}
\end{itemize}

A few hours after dissociation, neuritis start to elongate from the cell 
body and the first connections are established: neurons begin to 
communicate, and a new 2-D network is created. There is no fidelity with 
the in vivo architecture, but some fundamental mechanisms are maintained. 
The main neurotrasmitters of the CNS (i.e. glutamate and GABA) are 
expressed after two weeks in culture.\\

In vitro neural networks since they are extensively employed in research 
due to their fair level of organization and to the capability to easily
record their activity with arrays of electrodes. Nonetheless, some issues 
related to
\textit{in vitro} networks are the fact that they have no fidelity with 
respect to \textit{in vivo} neurons, starting from the fact that they 
generally arranged in 2D layers, instead of more complex three-dimensional 
networks.
However, new in vitro approaches were recently developed, for example 3D 
structures like \textbf{brain organoids or spheroids}. This kind of models 
can be obtained also with human derived stem cells that can be 
differentiated into neurons. This allows to reduce the involvement of 
animals and personalize the treatments for each patient.
\begin{figure}[H]
    \includegraphics[scale=0.2]{1_4}
    \centering
\end{figure}
There is also another type of approach that we can use for in vitro 
measurements, which is the one based on \textbf{brain slices}. In this 
model, brain slices maintain the same architecture of the brain from which 
they derive. The brain is glued onto the vibratome plate. The cut is 
performed in a very cold (0-2°C) and oxygenated (0.95 of o2 - 0.05 of CO2) 
physiological solution. From this method we record mostly field 
potentials.\\

Another crucial concept is the one of brain modularity, as a matter of 
fact the brain is redundant and intrinsically modular, due to the fact 
that it is composed of local networks that are embedded into networks of 
networks.\\

\subsection{In Vivo Systems}
This kind of procedure is more complex: people must be trained to perform 
it, and a specific environment is required. In this case, in fact, a real 
surgery is realized. 
Electrodes used are mostly intracortical electrodes. The same experimental 
set ups used for the in vitro analysis can be also used for the in vivo 
one.\\
\textbf{Calcium imaging} is another frequent technique, which finds the 
variations of calcium in the tissue under the microscope. It can be used 
also for interconnected cells. 

\subsection{From Raw Data to Point Process}
The process that we are more interested in performing is the detection of 
spikes. To do that, there is a specific procedure to follow, which is 
called \textbf{spike detection}.
Spike trains are derived by filtering the raw signal recorded from 
electrodes and by detecting the spikes present in it. Notice that the 
crucial point is not the magnitude of a certain spike, rather its position 
on the time axis, indicating when it was fired.
\begin{figure}[H]
    \includegraphics[scale=0.375]{1_5}
    \centering
\end{figure}
It is important to point out that the shape of a spike is highly 
influenced by the position of the measuring electrode w.r.t. the neuron 
that emitted it, enabling the researchers to recognize all the spikes 
emitted by the same source - i.e. a
particular neuron - and this is called spike sorting.\\


Simultaneous recording of multiple neurons offers new premises for 
investigating fundamental questions about brain functions, provided that 
the challenging problem of analyzing multiple simultaneously recorded 
spike trains can be properly addressed.
There are some working assumptions that are usually done when studying 
spike trains:
\begin{itemize}
    \item There is an enormous wealth of information about the structure 
and function of the nervous system which can be derived from careful study 
of the detailed timings of spike events. 
    \item The analysis of these signals can shed light on mechanisms of 
spike production within the observed cell (on the pre-synaptic input) and 
the mechanisms by which the spike is transformed into a post-synaptic 
output.
    \item Observation of multiple units can reveal details of 
interconnections and functional interactions.
    \item Neuronal processes at all levels involve a probabilistic element 
which must be adequately incorporated. 
\end{itemize}

This data present new analysis challenges because most standard signal 
processing techniques are designed primarily for continuous-valued data 
and not point processes. 
The main areas of research in this field are:
\begin{itemize}
    \item Identification and classification of spike events: spike 
detection and spike sorting. This latter technique allows us to identify 
the actual sources that caused the spikes.
    \item Techniques for measuring the "association" (cross-correlation, 
synchrony, etc...) between neural spike trains: e.g. cross-correlograms 
(both in time- and frequency-domain) 
    \item Quantification of the neural response to a stimulation: e.g. 
Post-Stimulus Time Histogram (PSTH), Joint- PSTH (JPSTH) 
\end{itemize}

Finally, let's highlight that the high number of electrodes employed in 
today's research implies a number of new challenges concerning data 
aquisition, storage, and analysis.\\
Since current techniques for spike train analysis are usually designed to 
analyze - at most - pairs of neurons, the challenge is to design methods 
that truly allow neuroscientists to perform multivariate analyses of 
multiple spike train data.
There are many benefits of developing such methods:
\begin{itemize}
    \item Easier quantification of reliability and statistical 
significance of experimental findings.
    \item Easier correlation between neural ensemble dynamics and behavior 
or relevant biological stimuli.
    \item Possibility to design more complex experimental procedures to 
investigate more relevant and important questions.
\end{itemize}

\subsection{Spike Trains}
A neuronal spike train is the sequence of nerve impulses or action 
potentials, produced by a neuron, typically observed with either 
intracellular or extracellular microelectrodes over a relatively long 
period of time. \\
Spike train analysis differs from 'classical' electrophysiological methods 
since the raw data of interest are not precise voltage measurements but 
rather \textbf{precise measurements of times of occurrence of events}. In 
fact, because of our 'all-or-one' conception of the nerve pulse, each 
spike is regarded as indistinguishable from the others produced by the 
same neuron. Additionally, with each spike can be associated a unique 
instant of time, e.g. the time of maximum excursion of electrical 
potential, which can be measured with high degree of precision. \\
The most important properties of individual spike events are two:

\begin{itemize}
    \item \textbf{Indistinguishability} (i.e. They are distinguishable 
only by where they occur in time)
    \item \textbf{Instantaneity} 
\end{itemize}

In probability theory and statistics, a time series of discrete events, 
such as a spike train, is called a \textbf{point process}. More 
specifically, being the spike train characterized by a certain degree of 
randomness and variability, we call it a stochastic point process which 
occurs in one dimension corresponding to the time axis. \\
Ensembles of spike trains from simultaneously recorded neurons are multi- 
dimensional point-process time series. These time series are both dynamic 
and stochastic. 

\begin{itemize}
    \item Dynamic means that their properties change through time
    \item Stochastic means that they change in a manner that can often be 
characterized by a probability model describing the likelihood of spikes 
at a given time. 
\end{itemize}

How can we mathematically describe this temporal series (i.e. the spike 
train)? 

\begin{itemize}
    \item A spike is an instantaneous event
    \item A spike is a pulse
    \item All the spikes have the same shape
\end{itemize}

Therefore, a single spike can be described in mathematical terms by using 
the Delta Dirac function:

\begin{equation*}
    ST(t) = \sum_{s=1}^N\delta(t-t_s)
\end{equation*}

For a network of M neurons, i.e. for simultaneously recorded spike trains 
from different neuronal sources, the mathematical description is given by:

\begin{equation*}
    ST_j(t) = \sum_{s=1}^{N_j}\delta(t-t_s) \hspace{1 cm} j=1, ..., M
\end{equation*}

 \begin{figure}[h]
    \centering
    \includegraphics[scale=0.3]{1_6.png}
\end{figure}
\newpage
