\subsection{Activation}
Doing a LFP analysis of our data, we have the possibility to look at \textbf{activation} phenomena. 
They represent how the cerebral cortex reacts to stimuli, that can be either electrical activities or exogenous stimuli like visual movements.\\
The simplest way of looking at EEG evoked response is the \textbf{Event Related Potential}.
When a stimulation is provided to the system, each trial contains a temporal mixture of signal and noise: signal is likely to be similar across trials, while noise fluctuates.\\
So, it would be better to analyze the data without considering noise. There are different ways to reduce its impact on the recording:
\begin{itemize}
    \item Averaging the amplitude of the signal across trials (and keeping the time dimension), noise cancels out and the ERP signals remain.
    \begin{figure}[H]
        \centering
        \includegraphics[scale=0.7]{13_1}
    \end{figure}
    \item Another way is to decompose the activity in a time frequency representation and then analyze what is the activation in terms of time-dynamics of the amplitude and how it changes across different frequency ranges.
    \begin{figure}[H]
        \centering
        \includegraphics[scale=0.7]{13_2}
    \end{figure}
\end{itemize}
These measures represent amplitude/power modulations, but what about phase? To understand the importance of the phase let's consider a simple image. 
\begin{figure}[H]
    \centering
    \includegraphics[scale=0.7]{13_3}
\end{figure}
The bigger figure is obtained from the smaller one after randomization of the amplitude spectrum while keeping the phase spectrum coherent.
This means that the spatial relationship of pixel luminance is contained in the phase rather than in the amplitude. 
Therefore, destroying the amplitude of a signal but keeping the phase intact there is still the possibility to reconstruct much of the information of the data.\\
So, considering a 1-D signal, i.e., looking at the time variations, the phase contains the dynamics response of the system more than just the amplitude.
\textbf{Phase coherence} (or synchronization) is a physiological mechanism to promote communication across brain network.\\
The capability to separate phase from amplitude is important to run analysis and make inferences.
One possibility to do it is using the so called \textbf{Inter Trial Phase Coherence} that measures the phase response of a neuronal ensemble to a certain stimuli.
In evoked activity there is the baseline activity and a slow amplitude response (in some of the channels). 
An analysis can be realized to understand if the event triggers at some point a coherent phase response at single sites in the network. 
To do it, the ERP can be used, but it only finds increases in amplitude, i.e., tells when the neurons that are below the electrode are synchronized (within the single event). The better option is the ITPC. It, in fact, is able to tell when the phase of the system resets the ongoing oscillations in response to a given stimulus. 
Practically speaking, the ITPC is the average across time of the amplitude, but measured as an average across trials of the phase. 
The difference with respect to ERP is that phase is a radiant value with periodicity, while amplitude is not. The averaging process is hence realized using the circular statistic properties and then measuring the resultant factor, which is the length of the vector of the averaged angles (their values are complex).
\begin{equation*}
    ITPC_{tf}=\biggl|n^{-1}\sum_{r=1}^ne^{ik_{tfr}}\biggr|
\end{equation*}
Computing this term for each time point, at the end a representation of the phase coherence for every single time point for a given channel across time is obtained.
Obviously, for a single time point a lot of angles are present, because a lot of trials are realized, as can be observed in the following image:
\begin{figure}[H]
    \centering
    \includegraphics[scale=0.5]{13_4}
\end{figure}
One vector for each trial is plotted and then the resultant is taken: if there is a preferred phase direction, the resultant vectors are not uniformly distributed on the unit cycle (\textit{Fig. A}), while the distribution tends towards this direction. On the other hand, if there is no causal effect on the event towards the activity (\textit{Fig. C}), the phase angles are uniformly distributed on the unit cycle, and the resultant vector will be very small.
\subsection{Connectivity}
Connectivity explains how different brain areas talk together, and it is based on a set of methods aimed to quantify interactions between brain regions. 
There are 3 ways to look at and characterize connectivity:
\subsubsection{Structural Connectivity}
\begin{figure}[H]
    \centering
    \subfigure[]{\includegraphics[width=0.2\textwidth]{13_5}} 
    \subfigure[]{\includegraphics[width=0.2\textwidth]{13_8}}
\end{figure}
It's the physical link between different brain regions, neurons or neuronal elements, due to the connections formed by axons one with the other.\\ 
It's relatively stable at shorter time scales (s or min), while at longer time scales (h or d) it is subject to significant morphological change and plasticity.\\
Currently only invasive tracing studies are capable of demonstrating direct axonal connections (you cut the brain and cut the axons). On the other hand, diffusion weighted imaging techniques, such as DTI, have an insufficient spatial resolution, but they are useful as whole brain in vivo markers of temporal changes in fibre tracts.
\subsubsection{Functional Connectivity}
It is a statistical concept that tells if two regions are more likely to share information. It captures deviations from statistical independence between distributed and often spatially remote neuronal units, regardless of whether they are connected by direct structural links.\\ 
It can be estimated in different ways, for instance measuring correlation or covariance, spectral coherence or phase-locking. Often connectivity matrices showing these kind of links are built. They are square matrices where on the axes the channels are present, that show the pairwise connectivity between two of them.
\begin{figure}[H]
    \centering
     \subfigure[]{\includegraphics[width=0.2\textwidth]{13_6}} 
    \subfigure[]{\includegraphics[width=0.2\textwidth]{13_9}}
\end{figure}
Functional connectivity is highly time-dependent on multiple time scales, from short to tens of hundreds of milliseconds, but it doesn't make reference to specific directional effects or to an underlying structural model.
\subsubsection{Effective Connectivity}
\begin{figure}[H]
    \centering
    \subfigure[]{\includegraphics[width=0.2\textwidth]{13_7}} 
    \subfigure[]{\includegraphics[width=0.2\textwidth]{13_10}}
\end{figure}
It may be viewed as the union of structural and functional connectivity: it describes networks of directional and causal effects (it is not a statistical property). It's mostly measured by intervention protocols: causal effects can, in fact, be inferred through systematic perturbations of the system, or, since causes must precede effects in time, through time series analysis.
One possible way is to look at the elicited response across the whole network, given that stimulations are applied to all the electrodes.
\subsubsection{Integration and Segregation}
\begin{figure}[H]
    \centering
    \includegraphics[scale=0.5]{13_11}
\end{figure}
Brain network is fast evolving in time and slowly changing in structure connections. 
There is a balance in the brain between \textbf{functional integration} and \textbf{functional segregation}, so that there are regions that are tightly connected together and modules where the activity of brain regions is more correlated are created. 
This modules contain a single way out, that are the \textbf{hubs} (graph theory). 
Through these hubs, the single module communicate with other parts of the network.\\
Connectivity allows to understand which region belongs to which module and which region is mostly used to communicate between modules.

\subsection{Functional Correlation}
Every analysis related to functional correlation will produce a matrix that shows that brain is not fully connected, but there are stronger and weaker links.
To study \textit{spontaneous activity} (not evoked), some formulations can be introduced.
\subsubsection{Spectral coherency}
Spectral coherency is defined as:
\begin{equation*}
C_{xy} = \frac{s_{xy}(f)}{\sqrt{s_{xx}(f)s_{yy}(f)}}
\end{equation*}
where \(s\) is the Fourier spectrum (fft) of the signal.
Spectral coherence is computationally expensive, so to use it, it's better to first down-sample the data.\\
NB: looking at the connectivity from a spike point of view, down-sampling is not possible: the only option is using a sparse matrix.\\
Since \(C_{xy}\) is a complex value, there two possible ways to look at it:
\begin{itemize}
    \item \textbf{Coherence}:
    \begin{equation*}
        Coh_{xy}=|C_{xy}|
    \end{equation*}
    It is a number comprised between 0 and 1, where 1 means complete coherence and 0 means no coherence. Of course, 0 never happens. It represents the strenght of the connection between the channels.
    \item \textbf{Imaginary Coherence}:
    \begin{equation*}
        iCoh_{xy}=|\Im{C_{xy}}|
    \end{equation*}
    It is important because volume conduction is modelled as zero phase synchronization, which only affects the real part of the spectrum. So, by looking at just imaginary part the zero lag synchronization can be removed.
\end{itemize}
\subsubsection{Phase Coupling}
A problem of \(C_{xy}\) is that it includes both info on phase synchronization and amplitude correlation. However, to better analyze either the phase or the amplitude, the signal can be decomposed using a Morlet transform. \\
From the decomposition,the phase synchronization or \textbf{phase locking value} can be obtained:
\begin{equation*}
    PLV_{xy}=\biggl|\frac{1}{n}\sum_{t=1}^ne^{j(\phi_x(t)-\phi_y(t)}\biggr|
\end{equation*}
This measure is very similar to the ITPC, even if the PLV is an average across time, whereas the ITPS is across trials. However their meaning is the same: if two channels have a constant time relation (lag), this is reflected in a constant phase difference measured as the resultant vector of the phase distribution along the unit cycle of all the possible phase angles.\\ 
So, looking at the time course, there can be a high PLV when the two activities are well synchronized, and a low PLV, when they are very different.
\begin{figure}[H]
    \centering
    \includegraphics[scale=1]{13_12}
\end{figure}

\subsection{How to assess significance of coupling?}
At this point, it is important to understand if the obtained results are significant or not.\\
A classical statistical analysis cannot be done, because every possible measure will be inflated by an error of type 2. Moreover, given that signals are natural, there is no certainty about with which distribution they can be represented.\\
So, a different way should be found. Usually, a \textbf{surrogate approach} is applied: a surrogate data is created, that maintains the auto-correlation of the signals in time, but breaks the cross-correlation among the channels. For instance, to test whether a value obtained using PLV makes sense, a piece of the surrogate data can be taken, broken into two and flipped in the time axis. In this way, the rows of the surrogate matrix still have the same number of samples, but they are shuffled: the cross-correlation among channels is disrupted, but the auto-correlation within a single channel is maintained. This process is then repeated many times.\\
Then with surrogate distribution we can conduct a classical hypothesis test (more in next lesson).
\begin{figure}[H]
    \centering
    \includegraphics[scale=0.8]{13_13}
\end{figure}






 