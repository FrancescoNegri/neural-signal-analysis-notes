\subsection{Introduction}
The burst analysis procedures generate a series of parameters representative for
each burst. The most commonly employed parameters are listed in the following:
\begin{itemize}
    \item Percentage of bursting channels
    \item Total number of bursts
    \item Mean bursting rate (\(MBR\), expressed in \(bursts/min\))
    \item Burst duration
    \item Inter burst interval (\(IBI\)) - it can be either start-to-start (not
    influenced by burst duration) or end-to-start
    \item Total number of spikes in a bursting channel
    \item Percentage of random spikes
    \item Mean intra-burst frequency
    \item Peak intra-burst frequency
\end{itemize}
Notice that sometimes it might be useful also to evaluate the out-burst spikes.

\subsection{Data visualization tools}
In the following there is a detailed description of the main tools to characterize and
visualize bursting activity.

\subsubsection{Inter Burst Interval}
The inter burst interval (\(IBI\)) represents the burst equivalent of \(ISI\) for
the spikes. Also in the case of \(IBI\) it is possible to plot an histogram,
denominated \(IBIH\). Burst are defined as macro events with a specific duration,
however it becomes less and less relevant as the recording period is increased.
Thus, under the hypothesis of long term recordings (in the order of tens of minutes)
it is possible to consider burst events as single-point events, exactly as it is done
for spikes. The mathematical formulation of the inter burst interval is given below:
\begin{align*}
    IBIH(\tau)=\frac{1}{M-1}\sum_{b=1}^{M-1}\delta(t_{b+1}-t_{b}-\tau)
\end{align*}
with \(M\) being the total number of burst events.
