\subsection{Introduction}
The burst analysis procedures generate a series of parameters representative for
each burst. The most commonly employed parameters are:
\begin{itemize}
    \item Percentage of bursting channels
    \item Total number of bursts
    \item Mean bursting rate (\(MBR\), expressed in \(bursts/min\))
    \item Burst duration (\(msec\))
    \item Inter burst interval (\(IBI\)) - it can be either start-to-start (not
          influenced by burst duration) or end-to-start (influenced by burst duration)
    \item Total number of spikes in a bursting channel
    \item Percentage of random spikes
    \item Mean intra-burst frequency
    \item Peak intra-burst frequency (maximum frequency inside the burst)
\end{itemize}
Sometimes it might be useful also to evaluate the out-burst spikes.

\subsection{Data visualization tools}
In the following there is a detailed description of the main tools to characterize and
visualize bursting activity.

\subsubsection{Inter Burst Interval}
The inter burst interval (\(IBI\)) represents the burst equivalent of \(ISI\) for the 
spikes. Also in the case of \(IBI\) it is possible to plot an histogram, denominated \(IBIH\). 
Bursts are defined as macro-events with a specific duration, which, however, becomes less and 
less relevant as the recording period increases, so that in the case of long-term recordings 
(tens of minutes), it is possible to consider a burst as a single event.\\
The mathematical formulation of the inter burst interval is given below:
\begin{align*}
    IBIH(\tau)=\frac{1}{M-1}\sum_{b=1}^{M-1}\delta(t_{b+1}-t_{b}-\tau)
\end{align*}
with \(M\) being the total number of burst events.
In a similar way, the analogous concepts of JIBI (Joint IBI, first row of the figure) 
and CIBI (Cross IBI, second row of the figure) can be defined.
\begin{figure}[H]
    \includegraphics[scale=0.2]{6_1}
    \centering
\end{figure}

\subsubsection{Fano Factor and Coefficient of Variation}
The \textbf{Fano Factor (\(FF\)) dispersion} is a reliable measure of the variability of events
(either spikes or bursts) across a given population. The \(FF\) dispersion is
immediately obtained by computing the average number of events \(\langle{N(T)}\rangle\)
in a time interval \(T\), together with its corresponding variance
\(Var[N(T)]\):
\begin{equation*}
    FF(T)=\frac{Var\bigl[N(T)\bigr]}{\langle{N(T)}\rangle}
\end{equation*}
On the other hand, by replacing the variance with the standard deviation
\(\sigma\bigl[N(T)\bigr]\), one could compute the \textbf{Coefficient of Variation}
\(CV\), that is a little more used:
\begin{align*}
    CV(T)=\frac{\sigma\bigl[N(T)\bigr]}{\langle{N(T)}\rangle}
\end{align*}
Notice that the Coefficient of Variation defined here is exactly the same
measure defined some pages above, denoted as \(C_v\).

\subsubsection{State Space Plot}
This type of representation visualizes data points as function of burst duration
and burst number (or alternatively burst rate). State space plots are
especially valuable to compare the activity of different channels or different
subjects, as they tend to point out differences in the bursts distribution:
for instance, they might be used to show the different activity between two
subjects, one with a pathological condition and the other being totally healthy.
\begin{figure}[H]
    \includegraphics[scale=0.33]{6_2}
    \centering
\end{figure}